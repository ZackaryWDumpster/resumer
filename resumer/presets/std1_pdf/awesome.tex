%!TEX TS-program = xelatex
%!TEX encoding = UTF-8 Unicode
% Awesome CV LaTeX Template for CV/Resume
%
% This template has been downloaded from:
% https://github.com/posquit0/Awesome-CV
%
% Author:
% Claud D. Park <posquit0.bj@gmail.com>
% http://www.posquit0.com
%
% Template license:
% CC BY-SA 4.0 (https://creativecommons.org/licenses/by-sa/4.0/)
%
%-------------------------------------------------------------------------------
% CONFIGURATIONS
%-------------------------------------------------------------------------------
% A4 paper size by default, use 'letterpaper' for US letter
\documentclass[11pt, a4paper]{awesome-cv}

% Configure page margins with geometry
\geometry{left=1.4cm, top=.8cm, right=1.4cm, bottom=1.8cm, footskip=.5cm}

% Color for highlights
% Awesome Colors: awesome-emerald, awesome-skyblue, awesome-red, awesome-pink, awesome-orange
%                 awesome-nephritis, awesome-concrete, awesome-darknight
\colorlet{awesome}{awesome-emerald}

% Colors for text
% Uncomment if you would like to specify your own color
%\definecolor{darktext}{HTML}{414141}
%\definecolor{text}{HTML}{333333}
%\definecolor{graytext}{HTML}{5D5D5D}
%\definecolor{lighttext}{HTML}{999999}
%\definecolor{sectiondivider}{HTML}{5D5D5D}
$for(define_colors)$
\definecolor{$define_colors.name$}{HTML}{$define_colors.value$}
$endfor$

% Set false if you don't want to highlight section with awesome color
\setbool{acvSectionColorHighlight}{true}

% If you would like to change the social information separator from a pipe (|) to something else
\renewcommand{\acvHeaderSocialSep}{\quad\textbar\quad}

%-------------------------------------------------------------------------------
%	PERSONAL INFORMATION
%	Comment any of the lines below if they are not required
%-------------------------------------------------------------------------------
% Available options: circle|rectangle,edge/noedge,left/right
% \photo[rectangle,edge,right]{./examples/profile}

\name{$first_name$}{$last_name$}
\address{$address$}
\mobile{$phone_num$}
\email{$email$}
\github{$github$}
\linkedin{$linkedin$}

\begin{document}

% Print the header with above personal information
% Give optional argument to change alignment(C: center, L: left, R: right)
\makecvheader[C]

% Print the footer with 3 arguments(<left>, <center>, <right>)
% Leave any of these blank if they are not needed
\makecvfooter
  {\today}
  {$firsname$ $lastname$~~~·~~~Resume}
  {\thepage}


%-------------------------------------------------------------------------------
%	CV/RESUME CONTENT
%-------------------------------------------------------------------------------
\cvsection{Key Qualifications}
\begin{cventries}
\cventry{}{}{}{}{
  \begin{cvitems}
  $for(qualification)$
      \item{$qualification$}
      $endfor$
  \end{cvitems}
}
\end{cventries}
%-------------------------------------------------------------------------------
%	Experiences
%-------------------------------------------------------------------------------
\cvsection{Experience}

\begin{cventries}
$for(experience)$
%---------------------------------------------------------
  \cventry
    {$experience.position$} % Job title
    {$experience.organization$} % Organization
    {$experience.location$} % Location
    {$experience.datestart$ - $experience.dateend$} % Date(s)
    {
      \begin{cvitems} % Description(s) of tasks/responsibilities
        $for(experience.items)$
        \item{$experience.items$}
        $endfor$
      \end{cvitems}
    }
    \linebreak
$endfor$
\end{cventries}

%-------------------------------------------------------------------------------
%	Projects
%-------------------------------------------------------------------------------

\cvsection{Projects}

\begin{cventries}
$for(project)$
%---------------------------------------------------------
  
  \cventry
    {$project.position$} % Job title
    {$project.name$} % name
    {$project.type$} % type
    {$project.datestart$ - $project.dateend$} % Date(s)
    {
      \begin{cvitems} % Description(s) of tasks/responsibilities
        $for(project.items)$
        \item{$project.items$}
        $endfor$
      \end{cvitems}
    }
    \linebreak
$endfor$

\end{cventries}

%-------------------------------------------------------------------------------
%	Education
%-------------------------------------------------------------------------------
\cvsection{Education}

\begin{cventries}
$for(education)$
    \cventry
    {$education.degree$} % Degree
    {$education.school$} % Institution
    {$education.location$} % Location
    {$education.datestart$ - $education.dateend$} % Date(s)
    {
      \begin{cvitems} % Description(s) bullet points
        $for(education.other)$
        \item{$education.other$}
        $endfor$
      \end{cvitems}
    }
    \linebreak
$endfor$
\end{cventries}

\end{document}
